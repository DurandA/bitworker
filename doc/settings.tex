%%%%%%%%%%%%%%%%%%%%%%%%%%%%%%%%%%%%%%%%%%%%%%%%%%%%%%%%%%%%%%%%%%%%%%
%\parindent 1cm
%\parskip 0.2cm
%\topmargin 0cm
%\bottommargin 0cm
%\rightmargin 0cm
%\leftmargin 0cm
%\oddsidemargin 1cm
%\evensidemargin 0.5cm
%\textwidth 15cm
%\textheight 21cm
%\makeindex

%%%%%%%%%%%%%%%%%%%%%%%%%%%%%%%%%%%%%%%%%%%%%%%%%%%%%%%%%%%%%%%%%%%%%%
%\renewcommand{\familydefault}{\sfdefault}

%%%%%%%%%%%%%%%%%%%%%%%%%%%%%%%%%%%%%%%%%%%%%%%%%%%%%%%%%%%%%%%%%%%%%%
% where are figures
\graphicspath{
  {./}
  {figures/}
  }

%%%%%%%%%%%%%%%%%%%%%%%%%%%%%%%%%%%%%%%%%%%%%%%%%%%%%%%%%%%%%%%%%%%%%%
%hyperref settings
\hypersetup{pdfauthor=Thomas Rouvinez, 
            pdftitle=HL7 Exchange Module, 
            pdfsubject=YourBookSubjectHere,
            colorlinks=true,
            linkcolor=black}

%%%%%%%%%%%%%%%%%%%%%%%%%%%%%%%%%%%%%%%%%%%%%%%%%%%%%%%%%%%%%%%%%%%%%%
% redefine some colors
\definecolor{shadethmcolor}{rgb}{0.9412,.9412,1.0000} % 
\definecolor{shaderulecolor}{rgb}{0.1529,0.2510,0.5451} % RoyalBlue 
\definecolor{lightergray}{gray}{0.95}
\definecolor{palegreen}{rgb}{0.7148,0.9219,0.6797} %pale green

%%%%%%%%%%%%%%%%%%%%%%%%%%%%%%%%%%%%%%%%%%%%%%%%%%%%%%%%%%%%%%%%%%%%%%
% define the theorem environments
% 1. definition
\theoremstyle{break}
\shadecolor{shadethmcolor}
\def\theoremframecommand{% 
\psframebox[fillstyle=solid,fillcolor=shadethmcolor,linecolor=shaderulecolor]} 
\newshadedtheorem{definition}{Definition}[section]
% 2. remark
\theoremstyle{plain} 
\theoremheaderfont{\normalfont\footnotesize\bfseries} 
\theorembodyfont{\normalfont\footnotesize} 
\theoremsymbol{\ensuremath{\clubsuit}} 
\theoremseparator{.}
\theoremindent1cm 
\theoremprework{\smallskip} 
\theorempostwork{\smallskip} 
\newtheorem*{remark}{$\rightarrow$ Remark}
% 3. seobs (software engineering observation)
\theoremstyle{plain} 
\theoremheaderfont{\normalfont\bfseries}
\theorembodyfont{\normalfont\normalsize}  
\theoremsymbol{\ensuremath{\clubsuit}} 
\theoremindent0cm 
\theoremseparator{.} 
\theoremprework{\bigskip\hrule} 
\theorempostwork{\hrule\bigskip} 
\newtheorem{seobs}{Software Engineering Observation}[section]
% 4. program output
\theoremstyle{nonumberplain} 
\theoremindent0.5cm 
\theorembodyfont{\ttfamily\small}
\theoremindent0cm 
\theoremseparator{} 
\theoremprework{\bigskip\verb}
\theorempostwork{\bigskip} 
\shadecolor{shadethmcolor}
\def\theoremframecommand{% 
\psframebox[fillstyle=solid,fillcolor=palegreen,linecolor=palegreen]} 
\newshadedtheorem{programoutput}{}

%%%%%%%%%%%%%%%%%%%%%%%%%%%%%%%%%%%%%%%%%%%%%%%%%%%%%%%%%%%%%%%%%%%%%%
% settings for the listings

\definecolor{green}{RGB}{0,180,10}
\definecolor{dkgreen}{rgb}{0,0.6,0}
\definecolor{gray}{rgb}{0.5,0.5,0.5}
\definecolor{mauve}{rgb}{0.58,0,0.82}

\lstset{
		language=Java,                % the language of the code
  		basicstyle=\footnotesize\ttfamily, % Standardschrift
        numbers=left,               % Ort der Zeilennummern
        numberstyle=\tiny,          % Stil der Zeilennummern
        stepnumber=1,               % Abstand zwischen den Zeilennummern
        numbersep=8pt,              % Abstand der Nummern zum Text
        tabsize=2,                  % Groesse von Tabs
        extendedchars=true,         %
        breaklines=true,            % Zeilen werden Umgebrochen
        commentstyle=\color{dkgreen},
        keywordstyle=\color{mauve},
    	   frame=b,         
        stringstyle=\color{blue}\ttfamily, % Farbe der String
        showspaces=false,           % Leerzeichen anzeigen ?
        showtabs=false,             % Tabs anzeigen ?
        xleftmargin=17pt,
        framexleftmargin=17pt,
        framexrightmargin=5pt,
        framexbottommargin=4pt,
        %backgroundcolor=\color{lightgray},
        showstringspaces=false      % Leerzeichen in Strings anzeigen ?     
 }